\documentclass{comnets-thesis}
\thesis{Master's Thesis}
\firstexaminer{First examiner: Prof. Dr.-Ing. Timm-Giel}
\secondexaminer{Second examiner: ?}
\supervisor{Supervisor: ?}
\title{Lorem ipsum dolor sit amet, consectetuer adipiscing elit, sed diam nonummy nibh euismod} % English title goes here
\germantitle{Germanici lorem ipsum dolor sit amet, consectetuer adipiscing elit, sed diam nonummy nibh euismod} % German title here
\author{Jane Doe} % Your name goes here
\program{Computer Science M.Sc.} % Your study program goes here
\matno{123456} % Your matriculation number goes here
\date{January 1, 2023} % Submission deadline here
\addbibresource{./library.bib}
\abstracttext{
	Your abstract goes here.
}

\begin{document}
\maketitlepage

\chapter{Introduction}\label{chp:introduction}
The structure of this thesis is \emph{only a suggestion}!
The chapter titles, the structure, whether to print the table of tables and all these things \emph{are up to you}.
You can use this template as a mere starting point. 
Only the first three pages you should rather not touch because the Prüfungsamt requires e.g., the Declaration of Originality, as found by \cite{SuperSmartGuy}.

In the \texttt{comnets-thesis.cls} file, after the table of contents you can find that an empty page is added.
This is only useful if you have an uneven number of pages so that the first page number is on the left side when you print this work double-sided.
If you have written an even number anyway \emph{then delete those lines}!

Oh and this template uses \texttt{biber} not \texttt{bibtex} for bibliography parsing.

Best of luck for your thesis or project work! :)

\chapter{Background}\label{chp:background}
\chapter{Related Work}\label{chp:related}
\chapter{Model}\label{chp:model}
\chapter{Implementation}\label{chp:implementation}
\chapter{Analysis}\label{chp:analysis}
\chapter{Conclusion}\label{chp:conclusion}
\appendix
\chapter{Appendix}\label{chp:appendix}
\listoffigures
\listoftables
\printbibliography[heading=bibintoc]

\end{document}
